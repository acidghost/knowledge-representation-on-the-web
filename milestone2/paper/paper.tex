%%%%%%%%%%%%%%%%%%%%%%% file moduleX_template.tex %%%%%%%%%%%%%%%%%
%
%
% This is a template for creating your papers for the course KRW
% It is based on the standard Latex template for Springer publications
% but contains a suggestion for the structure and some content of the
% paper.
%
% Please adapt this document wherever needed.
%
% For more information about the required Latex Style check the document
% typeinst.pdf in the StyleFiles directory.
%
%%%%%%%%%%%%%%%%%%%%%%%%%%%%%%%%%%%%%%%%%%%%%%%%%%%%%%%%%%%%%%%%%%%%%%%%%


\documentclass[runningheads,a4paper]{../../StyleFiles/llncs}

\usepackage{url}
\usepackage{graphicx}
\usepackage{amssymb}
\usepackage{listings}
\lstset{language=SQL,morekeywords={and, some, exactly}}
\usepackage{caption}
\usepackage{subcaption}


\newcommand{\keywords}[1]{\par\addvspace\baselineskip
\noindent\keywordname\enspace\ignorespaces#1}

\begin{document}

\mainmatter  % start of an individual contribution

% first the title is needed
\title{Data- and Ontology Paper: Quality Ontology for Handicap Parking Spots around Venues in Amsterdam}

% a short form should be given in case it is too long for the running head
\titlerunning{Data- and Ontology Paper}

% the name(s) of the author(s) follow(s) next
%
% NB: Chinese authors should write their first names(s) in front of
% their surnames. This ensures that the names appear correctly in
% the running heads and the author index.
%
\author{Alivanistos, Dimitris. \\ Baez, Selene. \\ Jemmett, Andrea. }
%
\authorrunning{Alivanistos, Dimitris. \\ Baez, Selene. \\ Jemmett, Andrea.}
% (feature abused for this document to repeat the title also on left hand pages)

% the affiliations are given next; don't give your e-mail address
% unless you accept that it will be published
\institute{\url{d.alivanistos@student.vu.nl} \and \url{s.baezsantamaria@student.vu.nl} \and \url{a.jemmett@student.vu.nl}}

\maketitle


\begin{abstract}
The abstract should summarize the contents of the paper and should
contain at least 70 and at most 150 words. It should be written using the
\emph{abstract} environment.
\end{abstract}


\section{Introduction}
Amsterdam, a busy and tourist city as it is, offers a variety of events throughout the year. As such, a relevant topic related to the previous is the accessibility the city provides for handicap people to attend such events. In particular, this project focuses on providing these users with a way to find Parking Slots when attending events in different venues in Amsterdam.

Though information about handicap parking slots in Amsterdam is public through the Municipality, the datasets are not structured in a human readable manner. Moreover, the link to venues and events is not well organized and has not been addressed in a user-friendly manner. To the extent of our knowledge, the application that gets closest to this description is the official website for Accessibility in Amsterdam, which still fails to provide with an easily searchable tool for finding handicap parking slots given an Event and/or Venue.
%TODO link to download datasets and to website.

This paper is an extension of the previous paper on Milestone 1, where we set up the project to solve the aforementioned problem, linking datasets from events at Theatres, and Museums and Galleries, to the Handicap Parking Slots dataset. In this paper the overall goal is to improve on the knowledge graph created for the application by creating an ontology that is formal yet intuitive for its potential users.

In the following section we further motivate the paper by providing concrete use cases that lead to a list of higher level requirements to be fulfilled by our project. Later, we describe an inspiration paper that guided us through the design of our ontology. We continue to describe our ontology, highlighting the major improvements from the knowledge graph in Milestone 1, followed up by evaluating the proposed ontology. Finally, some discussion provides with insights regarding the main contributions and future improvements on this project.

\section{Use Case and Requirements}
%\textit{Describe the use-case you have in mind that motivates the need for the ontology. This could be the original use case from milestone 1, but then you should explain why the schema you defined previously falls short. What requirements should the ontology meet? What is its intended scope.}
In order to better illustrate the motivation for this project we look back to the use cases from Milestone 1 to evaluate how well the current knowledge graph meets their needs. Next, we ambitiously create a couple more use cases to extend the potential uses of the application. We finalize by providing with a list of three specific requirements an ontology must meet to satisfy the users needs.

\subsection{Previous use cases}
\begin{enumerate}
	% Planning in advance
	\item John is in a wheelchair and wants to go to an exhibition in Van Gogh Museum Friday night. The day before the show, John uses our application to find the parking slots closest to the museum and heads to the show confidently. 
	% Quick search once in the theater (on the fly)
	\item Peter broke his leg last month and his friends, as an attempt to cheer him up, invited him to come to a comedy show at the Comedy Cafe. He is quite late and quickly browses our application and finds an area with several parking slots. He drives in that direction hoping to find an available slot. 
	% Event manager
	\item Mary is an event organizer who wants to host an exhibition in a museum next month. She is looking for an accessible venue so she would like to see the distribution of parking slots around the main galleries in Amsterdam. She has a list of five places in mind and uses our application to explore and choose the best venue.
\end{enumerate}

\subsubsection{Reflection on previous knowledge graph} 
In the previous Milestone we showed that our application successfully integrates the Events and Handicap Parking Slots datasets, thus allowing users to look for parking slots close to venues in a time-saving way. However, some of the needs expressed in the previous use cases can be better met. For example, currently John and Peter would have to browse through a long list of events till they find the exhibition or show they are attending. A better ontology that subdivides events into types could help them navigate the data more easily. 

Furthermore, another notable shortcoming is that parking slots are only sorted in terms of distance to venues, ignoring other relevant facts like its size. For example, Peter might want to head to a close, yet large parking slot in order to increase its chances to find an available spot. Similarly, Mary might want to have information about the capacity of the parking slots depending on the size of the event she is planning.

\subsection{New use cases}
\begin{enumerate}
	% Search by interest
	\item Toby is a Theatre fan and he is interested in watching as many plays as possible. He is in a wheelchair and needs accessible parking spots to go to Theatres when he selects a play to watch. 
	% Search by area
	\item Chelsea has appointments once every week at the Femme Amsterdam Hospital. She would like to look for events nearby to attend after her sessions. 
\end{enumerate}

\subsection{Requirements}
\label{requirements}
Given the above, we abstract the users needs into a high level set of requirements. To keep the project within a manageable scope, we limit the requirements to three items, summarized in the following list.

\begin{enumerate}
	\item A user should be able to look for specific types of events or specific venues according to his/her interests.
	\item Information about the size of the parking slots should be readily available through search. 
	\item Events, Venues and/or Parking Slots within a certain area should be related to each other and they can be easily grouped. 
\end{enumerate}


\section{Related Work}
%\textit{What other ontologies did you find that cover a similar domain. To what extent can they be reused. Have others (e.g. other students) developed similar datasets/ontologies that feed/meet your requirements, where do they fall short? Papers to look for are published at eg. ISWC, ESWC, EKAW and FOIS, or the Journal of Web Semantics, and the Semantic Web Journal.}
While doing research to meet the previous requirements, we ran across a variety of ontologies that aim to cover similar domains as ours. Within the domain of information related to cities, we found the work by Komninos et al.\cite{komninos2015smart} to be particularly inspirational for designing this paper's ontology. With its project, the authors bring awareness to the fact that, regardless of the effort by developers to create applications for smart cities, the impact of those is still low. They attribute the problem to the lack of coordinated effort to create a standardized high quality ontology, and propose to unify the existent ontologies into the Smart City Ontology (SCO).   %Following a structure thinking, SCO consists of 10 superclasses which reflect the three main aspects of a smart city. 

We want to accentuate the importance of coordination among well-formed ontologies and so we designed our ontology with the thought in mind to fit the broader schema laid down by Komninos el al. Ours is a specific and relatively small ontology that focuses on Handicap Parking Slots, Venues and Events. Yet, its full of potential can be achieved when assembled together with ontologies regarding other aspects of a city. 


\section{Methodology}
%\textit{Describe the methodology you followed for constructing the ontology. How did you guarantee that the ontology meets the needs and requirements of the use case? Have you used any existing design patterns, partial ontologies, competency questions etc.}

Given the requirements from Section \ref{requirements}, and with the objective mentioned above, we design the ontology through an iterative process, revising at each step that the users needs were met while keeping the potential for linking to other ontologies open. We use Protege 5.0 as a tool for creating the ontology as well as for testing the inferences among the stated facts. 

Furthermore, we elaborate on the expressiveness of the ontology and exploit the power of the Web Ontology Language (OWL) to represent more complex aspects such as cardinalities and equivalences among existing and newly created Subclasses and Object Properties.

As stated in the previous Milestone, we incorporate other ontologies like \texttt{dbepedia} \footnote{The \texttt{dbepedia} namespace is defined at \url{http://dbpedia.org/ontology/}} for location management, and \texttt{geo} \footnote{The \texttt{geo} namespace is defined at \url{http://www.w3.org/2003/01/geo/wgs84_pos}} for latitude and longitude management. This time, we exploit the usage of dbepedia and make use of the equivalence between dbo:Place and dbo:Location. Further explanation of the benefits of this are explained in the following section.

\section{The Ontology}
%\textit{Systematically describe the ontology and highlight the (interesting) design choices you had to make. Were there things you wanted to represent but couldn't given the limitations on the KR language?\footnote{If you do not encounter any limitations, you should be more ambitious about the model!} How did you work around these limitations? (integrity constraints, additional rules?) Describe how the ontology meets the 5 star model for Linked Vocabulary Use.}
A comparison between the old knowledge graph and the newly created ontology is graphically depicted in Figure \ref{fig:ontology}. 

\begin{figure}[h] \centering
	\begin{subfigure}{.5\textwidth} \centering
		\includegraphics[width=.9\textwidth]{img/old_ontology.jpg}
		\label{fig:old_ontology}
	\end{subfigure}%
	\begin{subfigure}{.5\textwidth} \centering
		\includegraphics[width=.9\textwidth]{img/ontology.png}
		\label{fig:new_ontology}
	\end{subfigure}
	\caption{Difference between original ontology (left) and updated ontology (right).}
	\label{fig:ontology}
\end{figure}

Given the degree of variation between both ontologies, we divide the updates on Entities and Properties and give a more detailed explanation in the following subsections.

\subsection{Entities}
The main difference with the previous version of the ontology is that each of the three main classes are now subdivided into more specific subclasses. For example:

\begin{itemize}
	\item \textbf{Events}: Divided into Plays and Exhibitions
	\item \textbf{Venues}: Divided into Theatres and Museums
	\item \textbf{Parking Slots}: Divided into Small, Medium and Large Slots
\end{itemize}


- First two distinctions come from the datasets themselves. The third one we created based on size. Parking Slot has datatype property quantity, which serves a a classifier

\begin{lstlisting}[captionpos=b, caption=Definition of Large Slot a subclass of Parking Slot, label=lst:owl, basicstyle=\ttfamily\small,frame=bt]
(dbo:location exactly 1 Location)
and (info exactly 1 xsd:string)
and (quantity exactly 1 xsd:unsignedInt)
\end{lstlisting}

- Our Location class extends dbo:Location class in the use of Borough

\subsection{Properties}

- Restrictions on datatype properties. Cardinalities are explicit, so that they reflect that an Event has only one Venue, for example

- Introduce 'Event venue' which has subproperties 'Exhibition Venue' and 'Play Venue' depending on their range. Also introduce its inverse 'Venue Events'

- We create a subproperty of dbo:location to be 'venue location'. We restrict dbo:location to have domain Parking Slots while 'venue location' has domain Venues. Both have Location as range.

- Properties for Borough, City and Country


\section{Section 6 - The Dataset}
\textit{Describe how you recasted the dataset(s) of the previous assignment to the new ontology (perhaps you found other datasets that you want to add). What did it take to do this? What metadata have you published alongside the dataset, and how did you generate it? (Provenance, VOID, etc.) Describe how your data meeds the 5 star model for Linked Data.}


\section{Section 7 - Evaluation}
\textit{Evaluate the ontology and dataset with respect to each other (does the ontology really sanction only the intended inference) and with respect to the requirements and use case (extend your application?!). You may want to analyze the ontology using some quality criteria from the literature.}

\subsection{Comparison with the dataset}
%TODO insert queries

\subsection{Comparison with requirements}
\begin{enumerate}
	\item Search by interests: Initial division of events and Venues by personal interest.
	\item Detailed location information: Ability to search for Venues and Parking Slots in same Borough is simple.
	\item Detailed information about Parking Slots: Size is explicit
\end{enumerate}

\subsection{Evaluate according to five stars}
- Usage of other ontologies like dbo and geo
- Metadata ?

\section{Discussion}
\textit{Here you summarize the preceding sections, describe the lessons learnt and discuss future work.}



\bibliographystyle{plain}
\bibliography{mybib}

\end{document}
